\documentclass[12pt,a4paper]{article} %,twoside

\usepackage{array}
\usepackage{tabularx}
\usepackage{textcomp}
\usepackage{amsmath}
\usepackage{amsthm} 
\usepackage{amssymb} 
\usepackage[utf8]{inputenc}
\usepackage[english,russian]{babel}
\usepackage{graphicx}
\usepackage{float}
\usepackage{indentfirst}
\usepackage{wrapfig}
\usepackage{epstopdf}
\usepackage{color}
\usepackage{listings}


% Задаю параметры макета страницы, все поля по 2 см
\oddsidemargin=-13pt
\topmargin=-66pt %%
\headheight=12pt %%
\headsep=38pt
\textheight=732pt
\textwidth=484pt
\marginparsep=14pt
\marginparwidth=43pt
\footskip=14pt
\marginparpush=7pt %%
\hoffset=0pt %%
\voffset=0pt %%
%\paperwidth=597pt %%
%\paperheight=845pt %%

%New commands.
\newcounter{StepCounter}
\newcommand{\step}[1]{\noindent\addtocounter{StepCounter}{1}\arabic{StepCounter}. #1}
\newcommand{\src}[1]{\textcolor{red}{\lstinline$>#1$}}
\newcommand{\abs}[1]{\left | #1 \right |}
%размер табуляции (для красной строки) в начале каждого абзаца
\parindent=1.5cm

\renewcommand{\baselinestretch}{1.25}
\newfloat{scheme}{tb}{sch}

\sloppy

% Подавление висячих строк
\clubpenalty=10000
\widowpenalty=10000


\begin{document}

\include{title}    % вставка титульника
\newpage

%@m batchload("D:/1.wxm")@

\fontsize{14pt}{15pt}\selectfont
\normalsize
\section{Нелинейное уравнение}
\step{Отделим корень, построив график в Maxima}\\
	$f(x) = x^5 - x - 0.2 ; \quad x \in [0.9, 1.1]$\\ 
	%@(defun f(x) (- (expt x 5) x 0.2))@ 
	%@(defun df(x) (- (* 5 (expt x 4)) 1))@
	%@(defvar *x0* 1.2)@ @(defvar *e* 0.0001)@ @(defvar *c* 1.6)@ @(defvar *a* 0.4)@
	\src{plot2d(x^5 - x - 0.2, [x, 0.9, 1.1], [ylabel, "y"]);}

	\begin{figure} [hbt] 
		\centering
			\includegraphics{@(plot-tex "plot2d(x^5 - x - 0.2, [x, 0.9, 1.1], [ylabel, \"y\"]")" "1")@}
		%\caption{график}
	\end{figure}

	\noindentОдин из корней приближённо равен $x_0 = 1,05$.\\
	Найдём в Maxima производную данной функции.\\
	\src{diff(x^5 - x - 0.2, x);}\\
	\src{@m diff(x^5 - x - 0.2, x)@}\\
	


\step{Уточним корень методом Ньютона.}\\\\
	@(table2tex (result2table (newton #'f #'df *x* *e*)) :epsilon 5)@
\newpage
\step{Метод дихотомии}\\\\
	@(table2tex (result2table (dihotom #'f *a* *x* *e*)) :epsilon 5)@

\step{Метод ньютона, модифицированный}\\\\
	@(table2tex (result2table (newton- #'f #'df *x* *e*)) :epsilon 5)@

\step{Метод хорд}\\\\
	@(table2tex (result2table (hord #'f  *x* *c* *e*)) :epsilon 5)@

\step{Метод секущих}\\\\
	@(table2tex (result2table (sec #'f  *x* *c* *e*)) :epsilon 5)@

\step{Метод Стеффенсона}\\\\	
	@(table2tex (result2table (stephansan #'f *x* *e*)) :epsilon 5)@

\normalsize
	

	\noindentПроверка:\\
	\src{find_root(x^5 - x - 0.2 = 0, x, 0.9, 1.1);}\\ %@m fpprintprec:6@
	\src{@m x:find_root(x^5 - x - 0.2 = 0, x, 0.9, 1.1)@}

	\noindentОтвет: $x = @m x@$
\newpage
\section{Нелинейная система} \setcounter{StepCounter}{0}
		%@m f: lambda ([x, y], cos(x) - y + 0.5)@
		%@m g:lambda ([x, y], y - sqrt(x) - 1)@
		$\begin{cases}
			\cos x  - y = -0,5\\
		 	y  - \sqrt{x} = 1
		 \end{cases}$\\
	
		\step{Изобразим графики функций в одной системе координат для отделения корня.}\\
		%\begin{wrapfigure}{r}{8cm} % "l" or "r" for the side on the page. And the width parameter for the width of the image space.
		\begin{figure}[hbt]
			\centering
				\includegraphics[scale=1.0]{@m (plot-tex"plot2d([cos(x) + 0.5, 1 + sqrt(x)], [x, 0, 1])" "equation-plot")@}
			%\caption{Подпись к картинке}
			%\label{label}
		%\end{wrapfigure}
		\end{figure}
		\src{plot2d([cos(x) + 0.5, 1 + sqrt(x)], [x, 0, 1]);}\\
		По графику приближённо находим $x \approx 0.2$, $y \approx 1.4$.\\
		Используем эти значения в качестве нулевого приближения к решению.\\
		
		
		\step{Выпишем формулы для итерационного процесса.}\\
		 Для этого вычислим частные производные функций\\
		 $f(x,y) = @m f(x,y)@$\\
		 $g(x,y) = @m g(x,y)@$\\
		 Получим: $f_x'= - \sin x = @m fx(x,y)@$, $f_y' = -1 = @m fy(x, y)@$, $g_x = -\frac{1}{2 \sqrt{x}} = @m gx(x, y)@$, $g_y' = 1 = @m gy(x, y)@$\\
		 Матрица Якоби имеет вид:\\
		 $\begin{pmatrix}
			-\sin & -1  \\
			-\frac{1}{2 \sqrt{x}} & 1 
		 \end{pmatrix}$\\
@		$@m J(x,y)@$
		 Для нахождения $J^{-1}$ используем формулу\\
		 $J^{-1}= \frac{1}{\abs{J}} \begin{pmatrix} 1 &  1 \\ \frac{1}{2 \sqrt{x}}&-\sin x \end{pmatrix} \Rightarrow $
		 \begin{tabular}{ l l }
			$J_{11}^{-1} = \frac{1}{\abs{J}}  g_y' = -\frac{1}{\abs{J}}$&  $J_{12}^{-1} =\frac{1}{\abs{J}} $\\
			$J_{21}^{-1} = \frac{1}{2 \sqrt{x} \abs{J}}$& $J_{22}^{-1} = - \frac{\sin x}{\abs{J}}$
		 \end{tabular}\\
@		 $@m Ji(x, y)@$
		 $\begin{pmatrix}x\\y\end{pmatrix}^{n+1} = \begin{pmatrix}x\\y \end{pmatrix}^{n} - J^{-1} \begin{pmatrix}x\\y \end{pmatrix}^{n} \begin{pmatrix}f(x^n, y^n)\\ g(x^n. y^n) \end{pmatrix} \Rightarrow \begin{pmatrix}x^{n+1}\\y^{n+1} \end{pmatrix} = \begin{pmatrix}x^n\\y^n \end{pmatrix} - \begin{pmatrix} J_{11}^{-1} & J_{12}^{-1} \\ J_{21}^{-1} & J_{22}^{-1}\end{pmatrix} \begin{pmatrix}f\\g \end{pmatrix} \Rightarrow \begin{cases} x^{n+1} = x^n - J_{11}^{-1}f - J_{12}^{-1}g\\ y^{n+1} = y^n - J_{21}^{-1}f - J_{22}^{-1}g\end{cases} 
\begin{picture}(0, 20) \put(3,1){\line(-1,1){12}} \put(3,-1){\line(-1,-1){12}} \end{picture}$ расчётные формулы, где \begin{tabular}{l}$f=f(x^n, y^n)$ \\ $g=g(x^n, y^n)$ \end{tabular} 
\\$J^{-1} = J^{-1} \begin{pmatrix}x^n\\y^n \end{pmatrix}$\\
		 \newpage\noindentВ таблицу внесём промежуточные вычисления, т. е. $f$, $g$ , $\abs{J}$, $J_{ij}^{-1}$  и оценочные величины \\
		 $\sigma^{n+1} = max\left\{ \abs{\frac{x^{n+1} - x^n}{x^{n+1}}}; \abs{\frac{y^{n+1} - y^n}{y^{n+1}}} \right\}$\\
\par\noindent
		 Дальнейшие вычисления сведём в таблицу\\\\
		 \begin{tabular}{| c | c | c | c | c | c | c | c | c |}
i&	x(i)&	y(i)&	f&	g&	fx&	fy&	gx&	gy\\ \hline
0&	0,2&	1,4&	0,080066578&	-0,047213595&	-0,198669331&	-1&	-1,118033989&	1\\
1&	0,224950938&	1,475109592	&-0,00030454&	0,000819662&	-0,223058536&	-1&	-1,054207497 & 1\\
2&	0,225354238	& 1,474715092&	-7,92743E-08&	1,90391E-07&	-0,223451657&	-1&	-1,053263755&	1\\
\end{tabular}\\\\ \par
\noindent
\begin{tabular}{| c | c | c | c | c | c | c | c |}
i&	|J|&	J11&	J12&	J21&	J22&	x(i+1)&	y(i+1)\\ \hline
0&	-1,31670332&	-0,75947252&	-0,75947252&	-0,84911610&	0,15088389&	0,22495093&	1,47510959\\
0&	-1,27726603&	-0,78292225&	-0,78292225&	-0,82536250&	0,17463749&	0,22535423&	1,47471509\\
0&	-1,27671541&	-0,78325991&	-0,78325991&	-0,82497927&	0,17502072&	0,22535432&	1,47471499\\
\end{tabular}\\\\ \par
\noindent
\begin{tabular}{| c | c | c | c  c  c | }
i&	dx&	dy&	max&	&	$\varepsilon$\\ \hline
0&	0,110917242	&0,050917974&	0,110917242	&    >&	0,0005\\
1&	0,00178963&	0,000267509	& 0,00178963&	>&	0,0005\\
2&	3,86207E-07&	6,69431E-08&	3,86207E-07&	<&	0,0005\\
	\end{tabular}\\
		 
		 Проверка в Maxima\\
		 \src{solve([cos(x) + 0.5, 1 + sqrt(x)], [x,y]);}\\
		 \src{[x = 0,225354238, y = 1,474715092]}\\
		 
\par\noindent		 Ответ: $x = 0,225354$ $y =  1,474715$
\end{document}
